\PassOptionsToPackage{unicode=true}{hyperref} % options for packages loaded elsewhere
\PassOptionsToPackage{hyphens}{url}
\PassOptionsToPackage{dvipsnames,svgnames*,x11names*}{xcolor}
%
\documentclass[10pt,ignorenonframetext,]{beamer}
\usepackage{pgfpages}
\setbeamertemplate{caption}[numbered]
\setbeamertemplate{caption label separator}{: }
\setbeamercolor{caption name}{fg=normal text.fg}
\beamertemplatenavigationsymbolsempty
% Prevent slide breaks in the middle of a paragraph:
\widowpenalties 1 10000
\raggedbottom
\setbeamertemplate{part page}{
\centering
\begin{beamercolorbox}[sep=16pt,center]{part title}
  \usebeamerfont{part title}\insertpart\par
\end{beamercolorbox}
}
\setbeamertemplate{section page}{
\centering
\begin{beamercolorbox}[sep=12pt,center]{part title}
  \usebeamerfont{section title}\insertsection\par
\end{beamercolorbox}
}
\setbeamertemplate{subsection page}{
\centering
\begin{beamercolorbox}[sep=8pt,center]{part title}
  \usebeamerfont{subsection title}\insertsubsection\par
\end{beamercolorbox}
}
\AtBeginPart{
  \frame{\partpage}
}
\AtBeginSection{
  \ifbibliography
  \else
    \frame{\sectionpage}
  \fi
}
\AtBeginSubsection{
  \frame{\subsectionpage}
}
\usepackage{lmodern}
\usepackage{amssymb,amsmath}
\usepackage{ifxetex,ifluatex}
\usepackage{fixltx2e} % provides \textsubscript
\ifnum 0\ifxetex 1\fi\ifluatex 1\fi=0 % if pdftex
  \usepackage[T1]{fontenc}
  \usepackage[utf8]{inputenc}
  \usepackage{textcomp} % provides euro and other symbols
\else % if luatex or xelatex
  \usepackage{unicode-math}
  \defaultfontfeatures{Ligatures=TeX,Scale=MatchLowercase}
\fi
\usetheme[]{Singapore}
\usefonttheme{serif}
% use upquote if available, for straight quotes in verbatim environments
\IfFileExists{upquote.sty}{\usepackage{upquote}}{}
% use microtype if available
\IfFileExists{microtype.sty}{%
\usepackage[]{microtype}
\UseMicrotypeSet[protrusion]{basicmath} % disable protrusion for tt fonts
}{}
\IfFileExists{parskip.sty}{%
\usepackage{parskip}
}{% else
\setlength{\parindent}{0pt}
\setlength{\parskip}{6pt plus 2pt minus 1pt}
}
\usepackage{xcolor}
\usepackage{hyperref}
\hypersetup{
            pdftitle={Linear regresjon (enkel og multippel)},
            pdfauthor={Stefanie Muff, Institutt for matematiske fag},
            colorlinks=true,
            linkcolor=Maroon,
            filecolor=Maroon,
            citecolor=Blue,
            urlcolor=blue,
            breaklinks=true}
\urlstyle{same}  % don't use monospace font for urls
\newif\ifbibliography
\usepackage{graphicx,grffile}
\makeatletter
\def\maxwidth{\ifdim\Gin@nat@width>\linewidth\linewidth\else\Gin@nat@width\fi}
\def\maxheight{\ifdim\Gin@nat@height>\textheight\textheight\else\Gin@nat@height\fi}
\makeatother
% Scale images if necessary, so that they will not overflow the page
% margins by default, and it is still possible to overwrite the defaults
% using explicit options in \includegraphics[width, height, ...]{}
\setkeys{Gin}{width=\maxwidth,height=\maxheight,keepaspectratio}
\setlength{\emergencystretch}{3em}  % prevent overfull lines
\providecommand{\tightlist}{%
  \setlength{\itemsep}{0pt}\setlength{\parskip}{0pt}}
\setcounter{secnumdepth}{0}

% set default figure placement to htbp
\makeatletter
\def\fps@figure{htbp}
\makeatother

\usepackage{multicol}

\title{Linear regresjon (enkel og multippel)}
\providecommand{\subtitle}[1]{}
\subtitle{ISTx1003 Statistisk læring og Data Science}
\author{Stefanie Muff, Institutt for matematiske fag}
\date{November 1 og 5, 2021}

\begin{document}
\frame{\titlepage}

\begin{frame}{Plan for i dag}
\protect\hypertarget{plan-for-i-dag}{}

\(~\)

\begin{itemize}
\item
  Hvem er vi?
\item
  Statistisk læring og data science
\item
  De tre temaene i modulen:

  \begin{itemize}
  \tightlist
  \item
    regresjon
  \item
    klassifikasjon og
  \item
    klyngeananlyse
  \end{itemize}
\item
  Læringsressurser og pensum
\item
  Prosjektoppgaven og Blackboard-informasjon
\item
  Tema: regresjon - med enkel lineær regresjon
\end{itemize}

\end{frame}

\begin{frame}{Læringsmål (av modulen)}
\protect\hypertarget{luxe6ringsmuxe5l-av-modulen}{}

Etter du har gjennomført denne modulen skal du kunne:

\begin{itemize}
\item
  forstå når du kan bruke regresjon, klassifikasjon og klyngeananlyse
  til å løse et ingeniørproblem
\item
  kunne gjennomføre multippel lineær regresjon på et datasett
\item
  bruke logistisk regresjon og nærmeste nabo for å utføre en
  klassifikasjonsoppgave
\item
  bruke hierarkisk og \(k\)-means klyngeanalyse på et datasett, forstå
  begrepet avstandsmål
\item
  og kunne kommunisere resultatene fra regresjon/
  klassifikasjon/klyngeanalyse til medstudenter og ingeniører
\item
  bli en kritisk leser av resultater fra statistikk/maskinlæring/
  statistisk læring/data science/kunstig intelligens når disse
  rapporteres i media, og forstå om resultatene er realistiske ut fra
  informasjonen som gis
\item
  kunne besvare prosjektoppgaven på en god måte!
\end{itemize}

\end{frame}

\begin{frame}{Hva er statistisk læring og data science?}
\protect\hypertarget{hva-er-statistisk-luxe6ring-og-data-science}{}

Todo

\end{frame}

\begin{frame}{Prosjektoppgaven}
\protect\hypertarget{prosjektoppgaven}{}

\(~\)

\begin{itemize}
\tightlist
\item
  Vi ser hvor informasjonen ligger på Blackboard og hvordan melde seg på
  gruppe.
\end{itemize}

\(~\)

\begin{itemize}
\tightlist
\item
  Vi ser på prosjektoppgaven på \url{https://s.ntnu.no/isthub}.
\end{itemize}

\end{frame}

\begin{frame}{Læringsmål (i dag)}
\protect\hypertarget{luxe6ringsmuxe5l-i-dag}{}

\(~\)

\begin{itemize}
\tightlist
\item
  Du kan lage en modell for å forstå sammenhengen mellom en respons og
  en eller flere forklaringsvariabler.
\end{itemize}

\(~\)

\begin{itemize}
\tightlist
\item
  Du kan lage en modell for å predikere en respons fra en eller flere
  forklaringsvariabler.
\end{itemize}

\end{frame}

\begin{frame}{Læringsressurser}
\protect\hypertarget{luxe6ringsressurser}{}

\vspace{2mm}

Alle ressurser er tilgjengelig her:

\url{https://wiki.math.ntnu.no/istx1003/2021h/start}

\(~\)

Tema Regresjon:

\vspace{2mm}

\begin{itemize}
\item
  \textbf{Kompendium}: Regresjon (pdf og html, by Mette Langaas)
\item
  \textbf{Korte videoer}: (by Mette Langaas)

  \begin{itemize}
  \tightlist
  \item
    Multippel lineær regresjon: introduksjon (14:07 min)
  \item
    Multippel lineær regresjon: analyse av et datasett (15:20 min)
  \end{itemize}
\item
  Denne forelesningen
\item
  \textbf{Disse slides} med notater
\end{itemize}

\end{frame}

\begin{frame}{Regresjon -- motiverende eksempel}
\protect\hypertarget{regresjon-motiverende-eksempel}{}

\centering\tiny(Veiledet læring - vi kjenner responsen)

\vspace{2mm}

\flushleft
\normalsize

\begin{itemize}
\tightlist
\item
  Kropssfett er en viktig indikator for overvekt, men vanskelig å måle.
\end{itemize}

\vspace{2mm}

\textbf{Spørsmål:} Hvilke faktorer tillater præsis estimering av
kroppsfettet?

\vspace{2mm}

Vi undersøer 243 mannlige deltakere. Kroppsfett (\%), BMI og andre
forklaringsvariabler ble målet. Spredningsplott:

\begin{center}\includegraphics[width=1\linewidth]{1Regresjon_files/figure-beamer/motivating-1} \end{center}

\end{frame}

\begin{frame}

For en model for funker god for prediksjon trenger vi \emph{multippel
linear regresjon}. Men vi begynner med \emph{enkel linear regresjon}
(bare en forklaringsvariabel):

\begin{center}\includegraphics[width=0.7\linewidth]{1Regresjon_files/figure-beamer/motivating2-1} \end{center}

\end{frame}

\begin{frame}{Enkel linear regresjon}
\protect\hypertarget{enkel-linear-regresjon}{}

\(~\)

\begin{itemize}
\item
  En kontinuerlig respons variabel \(Y\)
\item
  Bare \emph{en forklaringsvariable} \(x_1\)
\item
  Relasjon mellom \(Y\) og \(x\) er antatt å være \emph{linear}.
\end{itemize}

\vspace{6mm}

Hvis den lineare relasjonen mellom \(Y\) og \(x\) er perfekt, så gjelder
\[y_i = \beta_0 + \beta_1 x_{1i}\ \] for alle \(i\). Men..

\end{frame}

\begin{frame}

Hvilken linje er best?

\begin{center}\includegraphics[width=0.6\linewidth]{1Regresjon_files/figure-beamer/motivating3-1} \end{center}

\end{frame}

\begin{frame}

\begin{block}{Enkel linear regresjon}

\(~\)

\begin{enumerate}
[a)]
\tightlist
\item
  Kan vi tilpasse den ``rette'' linje til dataene?
\end{enumerate}

\(~\)

\begin{center}\includegraphics[width=0.5\linewidth]{1Regresjon_files/figure-beamer/squares-1} \end{center}

\begin{itemize}
\tightlist
\item
  \(\hat{y}_i = \hat\beta_0 + \hat\beta_1x_{1i}\).
\item
  \(\hat{e}_i = \hat{y}_i - y\)
\item
  \(\hat\beta_0\) og \(\hat\beta_1\) velges slik at
  \[SSE = \sum_i \hat{e}_i^2\] minimeres.
\end{itemize}

\end{block}

\end{frame}

\begin{frame}

\begin{enumerate}
[a)]
\setcounter{enumi}{1}
\tightlist
\item
  Kan vi tolke linja? Hvor sikkert er jeg på \(\hat\beta_1\) og linja?
  Vi trenger antakelser, KI og hypothesetest.
\end{enumerate}

\(~\)

\begin{enumerate}
[a)]
\setcounter{enumi}{2}
\tightlist
\item
  Fremtidige presisjoner av predikert \(y\) (kroppsfett)?
\end{enumerate}

\(~\)

\end{frame}

\begin{frame}{Linear regresjon -- antakelser}
\protect\hypertarget{linear-regresjon-antakelser}{}

\[Y_i = \underbrace{\beta_0 + \beta_1 x_{i1}}_{\hat{y}_i} + e_i\] med
\[e_i \sim \textsf{N}(0,\sigma^2) \ .\]

\centering

\includegraphics[width=0.7\textwidth,height=\textheight]{regrAssumptions.jpg}

\end{frame}

\begin{frame}

\begin{block}{Do-it-yourself ``by hand''}

\vspace{6mm}

Her kan du finne de beste parametrene selv: \vspace{2mm}

You can do this here: \vspace{2mm}

\url{https://gallery.shinyapps.io/simple_regression/}

\end{block}

\end{frame}

\begin{frame}{Multippel linear regresjon}
\protect\hypertarget{multippel-linear-regresjon}{}

Nesten det samme some enkel linear regresjon, we bare summerer flere
forklaringsvariabler:

\[Y_i = \beta_0 + \beta_1 x_{1i} + \beta_2 x_{2i} + \ldots + \beta_p x_{pi} + e_i \ , \quad e_i \sim\mathsf{N}(0,\sigma^2) \ .\]
\(~\)

For eksempel:

\[\text{bodyfat}_i = \beta_0 + \beta_1 \text{BMI}_i + \beta_2 \text{age}_i + e_i \ .\]

\end{frame}

\begin{frame}{Regresjonsanalyse i fem steg}
\protect\hypertarget{regresjonsanalyse-i-fem-steg}{}

Vi skal bruke statmodels.api og statmodels.formula.api for lineær
regresjon:

\(~\)

\textbf{Steg 1}: Bli kjent med dataene ved å se på oppsummeringsmål og
ulike typer plott

\textbf{Steg 2}: Spesifiser en matematisk modell

\textbf{Steg 3}: Initialiser og tilpass modellen

\textbf{Steg 4}: Presenter resultater fra den tilpassede modellen

\textbf{Steg 5}: Evaluer om modellen passer til dataene

\end{frame}

\begin{frame}

\begin{block}{Steg 1: Bli kjent med dataene}

\(~\)

\begin{itemize}
\tightlist
\item
  Histogram og boksplott av forklaringsvariable(r) \((x_1,\ldots, x_p)\)
  og \(y\).
\end{itemize}

\end{block}

\end{frame}

\end{document}
